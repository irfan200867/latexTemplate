% PDF build: latex -> bibtex -> latex -> dvips -> ps2pdf
% 0. Document class, options, and adjusted line spread.

%\documentclass[paper=a4paper,fontsize=11pt,pagesize,parskip=half-,numbers=noendperiod,
%               captions=nooneline,abstract=true,twocolumn=false,twoside=true]{scrartcl}
\documentclass[paper=160mm:90mm,fontsize=12pt,pagesize,parskip=half-,numbers=noendperiod,
              captions=nooneline,abstract=true,twocolumn=false,landscape]{scrartcl}
\linespread{1.12}

\usepackage[T1]{fontenc}
\usepackage{scrlayer-scrpage}
\usepackage{lmodern,amsmath,amsfonts,amssymb,microtype,graphicx,color,xcolor,hyperref,tikz,multicol}
\usepackage[framemethod=TikZ]{mdframed}
\usepackage[bahasa]{babel}

% 1. Packages and options for adjusting the page structure
\usepackage{calc}
\usepackage[includeheadfoot,top=3.5mm,bottom=3.5mm,left=5.5mm,right=5.5mm,headsep=6.5mm,
           footskip=8.5mm]{geometry}
\usepackage{tocstyle}
\usepackage{lipsum}
\usepackage[square]{natbib}             % bibliography style package

\bibpunct{(}{)}{;}{a}{}{,}              % bibliography style format:
                                        %         author [year]

\setlength\columnsep{30pt}

% 2. Some definitions
\newcommand{\judul}{Slide Aku}
\newcommand{\penulis}{P.~E.~Nulis}
\newcommand{\afiliasi}{Afiliasi Organisasi}
\newcommand{\tanggal}{\today}

% 3. Page style
\pagestyle{scrheadings}
\clearscrheadfoot % clear head and foot
\setkomafont{pageheadfoot}{\normalfont\color{white}\sffamily} % setting for page head and foot
% optical vertical centering of page contents
\makeatletter
\renewcommand*{\@textbottom}{\vskip \z@ \@plus 1fil}
\newcommand*{\@texttop}{\vskip \z@ \@plus .5fil}
\addtolength{\parskip}{\z@ \@plus .25fil} % stretch parskip
\makeatother

\definecolor{mybgcolor}{rgb}{0.41,0.60,0.55}
\hypersetup{colorlinks=true,linkcolor=mybgcolor}

% 4. Definition of the page head.
\ihead{ % head left
    \hspace{-2mm} %
    \begin{tikzpicture}[remember picture,overlay]
    \node[xshift=\paperwidth/2,yshift=-\headheight](mybar) at (current page.north west)[rectangle,
    fill,inner sep=0pt,minimum width=\paperwidth,
    minimum height=2\headheight,top color=mybgcolor!64,bottom color=mybgcolor]{}; % bar
    \node[below of=mybar,yshift=3.3mm,rectangle,shade,
    inner sep=0pt,minimum width=160mm,minimum height=1.5mm,top color=black!50,bottom color=white]{}; % shadow
    \end{tikzpicture} %
\textbf{\judul}
}

% 5. Definition of the page foot
\ifoot{ % foot left
    \hspace{-2mm} %
    \begin{tikzpicture}[remember picture,overlay]
    \node[xshift=\paperwidth/2,yshift=\footheight/2] at
    (current page.south west)[rectangle,fill,inner sep=0pt,minimum width=\paperwidth,
    minimum height=\footheight,top color=mybgcolor!64,bottom color=mybgcolor]{}; % bar
    \end{tikzpicture}%
    \vskip -11pt
    {\footnotesize\bfseries\penulis\ \raisebox{0.2mm}{|}\ \afiliasi}
}
\ofoot{\vskip -11pt%
    {\footnotesize\bfseries\thepage}
}

% 6. Settings for the table of content
\AtBeginDocument{\renewcommand{\contentsname}{\large Daftar Isi}}% change name of toc
\AtBeginDocument{\renewcommand{\abstractname}{\large Abstrak}}% change name of abstract
\makeatletter
\newtocstyle[noonewithdot]{nodotnopagenumber}{ % define tocstyle without dots and page numbers
    \settocfeature{pagenumberbox}{\@gobble} %
}
\makeatother
\usetocstyle{nodotnopagenumber}

\hypersetup{colorlinks=true,allcolors=mybgcolor}

% Some new environment
% environment-1: definition
\newcounter{definition}[section]\setcounter{definition}{0}
\renewcommand{\thedefinition}{\arabic{section}.\arabic{definition}}
\newenvironment{definition}[1]{%
    \refstepcounter{definition}
    % Code for box design goes here.
    \mdfsetup{%
        frametitle={%
            \tikz[baseline=(current bounding box.east),outer sep=0pt]
            \node[anchor=east,rectangle,fill=mybgcolor!90]
            {\strut  \textcolor{white}{Definisi~\thedefinition:~#1}};},%
        innertopmargin=10pt,linecolor=blue!20,%
        linewidth=2pt,topline=true,%
        frametitleaboveskip=\dimexpr-\ht\strutbox\relax%
    }
    \begin{mdframed}[]\relax%
     }{%
    \end{mdframed}}

% environment-2: example
\newenvironment{example}[0]{%
    % Code for box design goes here.
    \mdfsetup{%
        frametitle={%
            \tikz[baseline=(current bounding box.east),outer sep=0pt]
            \node[anchor=east,rectangle,fill=blue!20]
            {\strut Contoh};},%
        innertopmargin=10pt,linecolor=blue!20,%
        linewidth=2pt,topline=true,%
        frametitleaboveskip=\dimexpr-\ht\strutbox\relax%
    }
    \begin{mdframed}[]\relax%
    }{%
\end{mdframed}}

% environment-3: assignment
\newenvironment{assignment}[0]{%
    % Code for box design goes here.
    \mdfsetup{%
        frametitle={%
            \tikz[baseline=(current bounding box.east),outer sep=0pt]
            \node[anchor=east,rectangle,fill=blue!20]
            {\strut TUGAS};},%
        innertopmargin=10pt,linecolor=blue!20,%
        linewidth=2pt,topline=true,%
        frametitleaboveskip=\dimexpr-\ht\strutbox\relax%
    }
    \begin{mdframed}[]\relax%
    }{%
\end{mdframed}}

\begin{document}

\begin{flushleft}
    {\Huge\sffamily\textcolor{mybgcolor}{\bfseries\judul}}

    \vspace{2.5 true cm}

    {\large\bfseries\itshape\penulis}

    {\large\bfseries\itshape\afiliasi}

    {\bfseries\tanggal}
\end{flushleft}

\thispagestyle{empty}

\vfill
\pagebreak

\begin{abstract}
\lipsum[7]
\end{abstract}

\vfill
\pagebreak

\tableofcontents

\vfill
\pagebreak

\section{Lorem ipsum}
\lipsum[1]

\begin{example}
Salah satu rumus yang sangat dikenal adalah

\begin{equation}
x_{1,2} = \frac{-b \pm \sqrt{b^2 - 4 a c}}{2 a}
\end{equation}
\end{example}

\vfill
\pagebreak

\section{Nam dui}
\lipsum[2]

\begin{figure}
    \centering
    \includegraphics[width=0.5\linewidth]{plot1}
    \caption{Plot $\sin x$.}
    \label{fig:plot1}
\end{figure}

\begin{itemize}
\item \textcolor{mybgcolor}{\bfseries Lorem ipsum}
\item \textcolor{blue}{\bfseries Nam dui}
\item \textcolor{magenta}{\bfseries Nulla}
\item \textbf{Quisque}
\end{itemize}

\begin{definition}{Lingkaran Besar}
    Lingkaran besar di dalam suatu bola adalah lingkaran berjejari sama dengan jejari bola dan titik pusatnya berimpit dengan titik pusat bola.
\end{definition}

\vfill
\pagebreak

\section{Nulla}
\lipsum[3]

\vfill
\pagebreak

\section{Quisque}
\lipsum[4-6] See~\cite{siess2000}.

\vfill
\pagebreak

\begin{center}
    \textbf{\large Comparation}
\end{center}

\begin{multicols}{2}

    \begin{itemize}
        \item Lorem ipsum dolor sit amet, consectetuer adipiscing elit.
        \item Ut purus elit, vestibulum ut, placerat ac, adipiscing vitae, felis.
        \item Curabitur dictum gravida  mauris.
    \end{itemize}

    \columnbreak

    \begin{itemize}
        \item Lorem ipsum dolor sit amet, consectetuer adipiscing elit.
        \item Ut purus elit, vestibulum ut, placerat ac, adipiscing vitae, felis.
        \item Curabitur dictum gravida  mauris.
    \end{itemize}

\end{multicols}

\vfill
\pagebreak

\section{Commodo}
\lipsum[7] See~\cite{stepien2002}.

\begin{assignment}
    Buatlah bagan lingkaran dan ilustrasi untuk penjelasan pengertian fungsi-fungsi trigonometri $ \sin $, $ \cos $, sec, dan cosec.
\end{assignment}

\bibliography{scripArt}
\bibliographystyle{apalike}

\end{document}